\section{An Effect System}
\subsection{Intro and Poisoning}

\begin{frame}{Section 4: An Effect System}

\textbf{Goal:}
\begin{itemize}
  \item Extend the type system to track not just \emph{what} a program computes,
  but also \emph{which operations} it may perform.
  \item Types become \textbf{dirty} — they carry information about effects.
\end{itemize}

\vspace{1em}
\textbf{Motivation:}
\begin{itemize}
  \item Distinguish between pure and effectful computations.
  \item Enable more modular reasoning about side effects.
  \item Support features like effect polymorphism and effect masking.
\end{itemize}

\end{frame}

\begin{frame}{The Poisoning Problem: Motivation}

\textbf{Issue:} A function that \emph{could be pure} may be marked as effectful due to how it is defined.

\vspace{1em}
\textbf{Cause:}
\begin{itemize}
  \item In Core Eff, functions are first-class and effectful computations are values.
  \item This means we can return functions from effectful contexts.
  \item However, if such a function is returned from within a computation that performs effects, the \emph{entire function} may be marked as dirty — even if it doesn’t perform any effects.
\end{itemize}

\vspace{1em}
\textbf{Consequence:}
\begin{itemize}
  \item The function type becomes \emph{poisoned} by unrelated effects.
  \item Makes type signatures overly conservative and imprecise.
\end{itemize}

\end{frame}

\begin{frame}{The Poisoning Problem:  Example}

\textbf{Assume:}
\begin{itemize}
  \item A ground type \texttt{string}
  \item An effect instance \( \texttt{std} : \texttt{Console}^R \), with:
    \[
    \Sigma(\texttt{Console}) = \{ \texttt{write} : \texttt{string} \rightarrow \texttt{unit} \}
    \]
\end{itemize}

\vspace{1em}
\textbf{Consider:}
\begin{align*}
& \texttt{let ignore} = \texttt{val (fun msg} \mapsto \texttt{val ()}) \\
& \texttt{let f} = \texttt{if } b \texttt{ then (val ignore)} \\ 
& \texttt{else val (std\#write) in val ignore}
\end{align*}

\vspace{1em}
\textbf{What is the type of \texttt{ignore}?}
\begin{itemize}
  \item Ideally: \quad \( \texttt{string} \rightarrow \texttt{unit} \;!\; \emptyset \)
  \item But in the conditional: \( \texttt{string} \rightarrow \texttt{unit} \;!\; \{ \texttt{std}\#\texttt{write} \} \)
\end{itemize}
\textbf{Problem:} the \texttt{ignore} function is pure, but gets \textbf{over-typed} due to context.
\end{frame}

\begin{frame}{Solving Poisoning via Subtyping}

\textbf{Key idea:}
\begin{itemize}
  \item Allow a dirty type \( A\ !\ \Delta \) to be treated as \( A\ !\ \Delta' \) when \( \Delta' \subseteq \Delta \).
  \item This lets us safely discard “irrelevant” effects from a type when they are not used.
\end{itemize}

\vspace{1em}
\textbf{In the example:}
\[
\texttt{string} \rightarrow \texttt{unit} \;!\; \{ \texttt{std}\#\texttt{write} \}
\le
\texttt{string} \rightarrow \texttt{unit} \;!\; \emptyset
\]

\vspace{1em}
\textbf{Benefit:}
\begin{itemize}
  \item Avoids over-tainting pure expressions.
  \item Makes the type system more precise and modular.
\end{itemize}
\end{frame}

\begin{frame}{Dirty Type Subtyping Rule}

\textbf{Subtyping on dirty types:}
\[
\frac{A = A' \quad \Delta' \subseteq \Delta}
{A\ !\ \Delta \leq A'\ !\ \Delta'}
\]

\vspace{1em}
\textbf{This rule allows:}
\begin{itemize}
  \item Weakening the dirt component of a type
  \item No change in the result type \( A \)
\end{itemize}

\vspace{1em}
\textbf{Notation:} \( A\ !\ \Delta \leq A\ !\ \Delta' \)  
means: we can treat a computation that may perform \( \Delta \) as one that may perform any smaller set \( \Delta' \)
\end{frame}

\begin{frame}{Core Eff Syntax: Types (with Dirt)}

\textbf{Value types} \( A, B \) and \textbf{Computation types} \( C \) are defined as:

\begin{align*}
A, B ::= &\quad \alpha \quad \text{(type variable)} \\
        &\mid \texttt{bool} \mid \texttt{nat} \mid \texttt{string} \mid \dots \\
        &\mid A \rightarrow C \quad \text{(function type)} \\
        &\mid E^R \quad \text{(effect instance of type } E \text{ at region } R) \\
        &\mid \texttt{handler } C \Rightarrow D \quad \text{(handler type)}
\end{align*}

\vspace{0.5em}

\begin{align*}
C, D ::= A\ !\ \Delta \quad \text{(computation type with dirt)}
\end{align*}

\vspace{1em}

\textbf{Dirt} \( \Delta \): a finite set of operation labels of the form \( \iota \# \texttt{op} \)

\end{frame}

\begin{frame}{Effect Signatures, Regions, and Dirt}

\textbf{Effect Signature \( \Sigma \):}
\[
\Sigma(E) = \{ \texttt{op}_i : A_i \rightarrow B_i \}_{i \in I}
\quad \text{for each effect name } E
\]

\vspace{0.5em}
\textbf{Effect Instances:}
\[
\iota : E^R
\quad \text{means } \iota \text{ is an instance of effect } E \text{ in region } R
\]

\vspace{1em}
\textbf{Operation Labels:}
\[
\iota \# \texttt{op}
\quad \text{identifies the operation } \texttt{op} \text{ performed on instance } \iota
\]

\vspace{1em}
\textbf{Dirt:}
\[
\Delta \subseteq \{ \iota \# \texttt{op} \}
\quad \text{is a finite set of operation labels}
\]

\vspace{1em}
\textbf{Dirty type:}
\[
A\ !\ \Delta \quad \text{means: computation returning } A \text{ may perform effects } \Delta
\]

\end{frame}
\subsection{Type system}
\begin{frame}{Typing Judgments and Contexts}

\textbf{Typing judgments:}
$$
\Gamma \tvdash e : A
\qquad
\Gamma \tvdash c : A\ !\ \Delta
$$

\begin{itemize}
  \item \( e \): a pure expression, producing a value of type \( A \)
  \item \( c \): a computation, producing a result of type \( A \), possibly performing effects \( \Delta \)
\end{itemize}

\vspace{1em}
\textbf{Typing context \( \Gamma \):}
\begin{itemize}
  \item A finite set of assumptions of the form:
    \begin{itemize}
      \item \( x : A \) — term variables
      \item \( \alpha \) — type variables
      \item \( R \) — region variables
    \end{itemize}
\end{itemize}

\vspace{1em}
\textbf{Note:} We write \( \Gamma, x : A \tvdash e : B \) to extend contexts.
\end{frame}

\begin{frame}{Typing Rules: Constants and Variables}

\textbf{Pure value judgments:}
\[
\Gamma \tvdash e : A
\]

\vspace{1em}

\[
\frac{c : \texttt{const of type } A}
{\Gamma \tvdash c : A}
\quad\textsc{(Const)}
\]

\vspace{1em}

\[
\frac{x : A \in \Gamma}
{\Gamma \tvdash x : A}
\quad\textsc{(Var)}
\]

\end{frame}

\begin{frame}{Typing Rules: Functions and Applications}

\textbf{Judgment form:}
\[
\Gamma \tvdash e : A
\qquad
\Gamma \tvdash c : A\ !\ \Delta
\]

\vspace{1em}

\textbf{Function abstraction:}
\[
\frac{\Gamma, x : A \tvdash c : B\ !\ \Delta}
{\Gamma \tvdash \texttt{fun } x \mapsto c : A \rightarrow B\ !\ \Delta}
\quad\textsc{(Fun)}
\]

\vspace{1.5em}

\textbf{Function application:}
\[
\frac{
  \Gamma \tvdash v_1 : A \rightarrow B\ !\ \Delta
  \quad
  \Gamma \tvdash v_2 : A
}{
  \Gamma \tvdash v_1\ v_2 : B\ !\ \Delta
}
\quad\textsc{(App)}
\]

\end{frame}

\begin{frame}{Typing Rule: \texttt{val} (Subtyping-Safe)}

\textbf{Rule:}
\[
\frac{\Gamma \tvdash e : A}
{\Gamma \tvdash \texttt{val } e : A\ !\ \emptyset}
\quad\textsc{(Val)}
\]

\vspace{1em}
\textbf{Key Point:}
\begin{itemize}
  \item The `val` construct lifts a pure expression into a pure computation.
  \item The resulting computation has dirt \( \emptyset \).
\end{itemize}

\vspace{1em}
\textbf{Why this is general:}
\begin{itemize}
  \item Due to subtyping:
  \[
  A\ !\ \emptyset \leq A\ !\ \Delta \quad \text{for any } \Delta
  \]
  \item This allows us to type `val e` at any dirt level needed downstream.
\end{itemize}

\end{frame}

\begin{frame}{Typing Rule: \texttt{let}}

\textbf{Rule:}
\[
\frac{
  \Gamma \tvdash c_1 : A\ !\ \Delta_1
  \quad
  \Gamma, x : A \tvdash c_2 : B\ !\ \Delta_2
}{
  \Gamma \tvdash \texttt{let } x = c_1 \texttt{ in } c_2 : B\ !\ \Delta_1 \cup \Delta_2
}
\quad\textsc{(Let)}
\]

\vspace{1em}
\textbf{Interpretation:}
\begin{itemize}
  \item Run \( c_1 \), bind its result to \( x \), then run \( c_2 \).
  \item All effects from both computations are accumulated.
\end{itemize}

\end{frame}

\begin{frame}{Typing Rule: Sequencing \texttt{c₁ ; c₂}}

\textbf{Rule:}
\[
\frac{
  \Gamma \tvdash c_1 : A\ !\ \Delta_1
  \quad
  \Gamma \tvdash c_2 : B\ !\ \Delta_2
}{
  \Gamma \tvdash c_1\ ;\ c_2 : B\ !\ \Delta_1 \cup \Delta_2
}
\quad\textsc{(Seq)}
\]

\vspace{1em}
\textbf{Interpretation:}
\begin{itemize}
  \item Run \( c_1 \), discard its result, then run \( c_2 \).
  \item The resulting dirt includes effects from both computations.
\end{itemize}

\end{frame}

\begin{frame}{Typing Rule: Operation Call }

\textbf{Assume:}
\begin{itemize}
  \item \( \iota : E^R \in \Gamma \)
  \item \( \texttt{op} : A^{\texttt{op}} \rightarrow B^{\texttt{op}} \in \Sigma(E) \)
\end{itemize}

\vspace{1em}

\textbf{Rule:}
\[
\frac{
  \Gamma, y: B^{op}  \tvdash c : A^{\texttt{op}} \ !\ \Delta
}{
  \Gamma\tvdash \iota \# \texttt{op } c : B^{\texttt{op}} \ !\ \Delta \cup \{ \iota \# \texttt{op} \}
}
\quad\textsc{(Op)}
\]

\vspace{1em}

\textbf{Interpretation:}
\begin{itemize}
  \item The argument \( c \) is itself a computation, not just a pure expression.
  \item The dirt includes the dirt of \( c \), plus the label for the operation call.
  \item This allows for effects to be embedded in the argument to the operation.
\end{itemize}

\end{frame}

\begin{frame}{Typing Rule: Handler Application}

\textbf{Handler typing:}
\[
\Gamma \tvdash h : A\ !\ \Delta \Rightarrow B\ !\ \Delta'
\qquad
\Gamma \tvdash c : A\ !\ \Delta
\]

\textbf{Rule:}
\[
\frac{
  \Gamma \tvdash h : A\ !\ \Delta \Rightarrow B\ !\ \Delta'
  \quad
  \Gamma \tvdash c : A\ !\ \Delta
}{
  \Gamma \tvdash \texttt{with } h \texttt{ handle } c : B\ !\ \Delta'
}
\quad\textsc{(Handle)}
\]

\vspace{1em}

\textbf{Interpretation:}
\begin{itemize}
  \item Handler \( h \) handles all operations in \( \Delta \) performed by \( c \)
  \item Produces a computation of type \( B\ !\ \Delta' \)
  \item The dirt \( \Delta' \) reflects any effects left unhandled by \( h \)
\end{itemize}

\end{frame}

\begin{frame}{Handler Clause Type: \( C / \Delta \)}

\textbf{Notation:}
\[
C / \Delta
\quad \text{means: computation of type } C \text{ with handled dirt } \Delta
\]

\vspace{1em}

\textbf{Example:}
\[
A\ !\ \Delta \;/\; \Delta
\quad \text{means: computation of type } A\ !\ \Delta \text{ with all effects handled}
\]

\vspace{1em}

\textbf{Usage:}
\begin{itemize}
  \item Appears in the type of a handler:
  \[
  \Gamma \tvdash h : A\ !\ \Delta \Rightarrow B\ !\ \Delta'
  \]
  \item The \texttt{val}-clause and each \texttt{op}-clause must be typed against \( C / \Delta \)
  \item It expresses that all operations in \( \Delta \) are handled — and others must not appear
\end{itemize}

\end{frame}

\begin{frame}{Typing Rule: Handler Definition (\textsc{Hand})}

\textbf{Handler has the form:}
\[
\texttt{handler val } x \mapsto c_v \mid \texttt{op}_i\ x\ k \mapsto c_i
\]

\textbf{Rule:}
\[
\frac{
  \Gamma, x : A \tvdash c_v : B\ !\ \Delta'
  \quad
  \Gamma \tvdash \texttt{ocs} : B\ !\ \Delta' / \Delta'' 
  \quad \Delta \subset \Delta' \cup \Delta''
}{
  \Gamma \tvdash h : A\ !\ \Delta \Rightarrow B\ !\ \Delta'
}
\quad\textsc{(Hand)}
\]

\vspace{1em}

\textbf{Interpretation:}
\begin{itemize}
  \item The \texttt{val}-clause transforms results of type \( A \) into computations of type \( B\ !\ \Delta' \)
  \item The op-cases collectively handle all operations in \( \Delta \)
  \item The resulting handler maps \( A\ !\ \Delta \) to \( B\ !\ \Delta' \)
\end{itemize}

\end{frame}

\begin{frame}{Typing Rule: Op-Cases (Empty) — \textsc{OpCases-Nil}}

\textbf{Rule:}
\[
\frac{}{\Gamma \tvdash nil_{C} : A\ !\ C / \emptyset  }
\quad\textsc{(OpCases-Nil)}
\]

\vspace{1em}

\textbf{Interpretation:}
\begin{itemize}
  \item An empty handler (no op-cases) can handle a computation with no effects
  \item This is the base case for building op-cases recursively
  \item The result dirt \( \Delta' \) can still be non-empty — it's from the `val` clause
\end{itemize}

\end{frame}

\begin{frame}{Typing Rule: Op-Cases (Extend) — \textsc{OpCases-Cons}}

\textbf{Exact rule from the paper:}

\[
\frac{
  \begin{array}{l}
    \iota : E^R \in \Gamma \quad\quad \texttt{op} : A^{\texttt{op}} \rightarrow B^{\texttt{op}} \in \Sigma(E) \\
    \Gamma, x : A^{\texttt{op}},\ k : B^{\texttt{op}} \rightarrow C\ !\ \Delta''
    \tvdash c : C\ !\ \Delta'' \\
    \Gamma \tvdash \texttt{ocs} : C\ !\ \Delta''  / \Delta' 
    \quad
    \Delta \subset \Delta' \cup^. R \#op
  \end{array}
}{
  \Gamma \tvdash (\iota \# \texttt{op } x\ k \mapsto c) \mid \texttt{ocs}
  : C !\ \Delta'' / \Delta 
}
\quad\textsc{(OpCases-Cons)}
\]

\end{frame}