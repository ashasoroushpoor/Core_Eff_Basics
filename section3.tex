\section{Operational Semantics Of Core-Eff}

\begin{frame}{Operational Semantics: Values and \texttt{val}}
\textbf{Values:}
\[
v ::= x \mid \texttt{fun } x \Rightarrow c \mid \texttt{handler } \cdots
\]
\begin{itemize}
  \item Values are fully evaluated and are not reducible.
\end{itemize}

\vspace{1em}
\textbf{Computation form:}
\[
\texttt{val } v
\]
\begin{itemize}
  \item A computation that simply returns value \( v \).
  \item This is a \emph{terminal configuration} under the small-step semantics.
  \item There is \textbf{no reduction rule} for \texttt{val v}, i.e., no \( \texttt{val } v \rightsquigarrow \cdots \)
\end{itemize}

\vspace{1em}
This is the base case for computations — the final result before effects or handlers are involved.
\end{frame}

\begin{frame}{Small-Step Semantics: \texttt{if} and \texttt{match}}

\textbf{Conditional branching:}
\[
\frac{}{\texttt{if true then } c_1 \texttt{ else } c_2 \rightsquigarrow c_1}
\]
\[
\frac{}{\texttt{if false then } c_1 \texttt{ else } c_2 \rightsquigarrow c_2}
\]

\vspace{1em}
\textbf{Pattern matching:}
\[
\frac{}{\texttt{match } 0 \texttt{ with } 0 \Rightarrow c_1 \mid S\,x \Rightarrow c_2 \mapsto c_1}
\]
\[
\frac{}{\texttt{match } S\,v \texttt{ with } 0 \Rightarrow c_1 \mid S\,x \Rightarrow c_2 \mapsto c_2[v/x]}
\]

\vspace{1em}
\textbf{Note:} \texttt{if} and \texttt{match} reduce only when their scrutinees are values.
\end{frame}

\begin{frame}{Small-Step Semantics: \texttt{fun}, \texttt{let}, and \texttt{with}}

\textbf{Function application:}
\[
\frac{}{(\texttt{fun } x \Rightarrow c)\ v \rightsquigarrow c[v/x]}
\]

\vspace{1em}
\textbf{Let-binding:}
\[
\frac{}{ \texttt{let } x = \texttt{val } v \texttt{ in } c \rightsquigarrow c[v/x] }
\]
\[
\frac{c_1 \rightsquigarrow c_1'}{\texttt{let } x = c_1 \texttt{ in } c_2 \rightsquigarrow \texttt{let } x = c_1' \texttt{ in } c_2}
\]

\vspace{1em}
\textbf{Handler congruence:}
\[
\frac{c \rightsquigarrow c'}{\texttt{with } e \texttt{ handle } c \rightsquigarrow \texttt{with } e \texttt{ handle } c'}
\]

% \vspace{1em}
% Only one subterm is reduced per step — all rules are in left-to-right evaluation order.
\end{frame}

\begin{frame}{Small-Step Semantics: Effect Invocation}
\textbf{Effect operation call:}
\[
\frac{}{
  \texttt{let } x = \iota \# \texttt{op } e \texttt{ in } c
  \rightsquigarrow
  \iota \# \texttt{op } e (y.\ \texttt{let } x = y \texttt{ in } c)
}
\]

\vspace{1.5em}
\textbf{Explanation:}
\pause
\begin{itemize}
  \item This rule models an \emph{unhandled effect}. So for the effect to be handled the operation must be propogated with this rule.
  \item It suspends the computation, exposing the continuation as a function of the result.
  \item The handler, if present, will later match and resume from this.
\end{itemize}
\end{frame}

% \begin{frame}{Small-Step Semantics: Handling Operations (Full Rule)}

% \textbf{Handler (operation case):}
% \[
% \frac{
%   \begin{array}{l}
%     h = \texttt{handler val } x \Rightarrow c_v \mid \texttt{op } x\ k \Rightarrow c_{op} \\
%     r = \iota \# \texttt{op } e\ (y.c) \\
%     \\
%     \texttt{with } h\ \texttt{handle } r \rightsquigarrow c_{op}
%     [e/x,\ (\texttt{fun } y \Rightarrow \texttt{with } h\ \texttt{handle } c)/k]
%   \end{array}
% }{
%   \texttt{with } h\ \texttt{handle } r \rightsquigarrow \cdots
% }
% \]

% \vspace{1em}
% This rule applies when an operation \texttt{op} from instance \(\iota\) is matched by handler \(h\),
% and constructs a resumption to continue the suspended computation.
% \end{frame}
\begin{frame}{Small-Step Semantics: Handler Rule }

\[
\frac{
  h = \texttt{handler val } x \mapsto c_v \mid ocs
  \qquad
  (op: A^{op} \rightarrow B^{op}) \in \Sigma_E
}{
\begin{aligned}
  & \texttt{with } h \texttt{ handle } (\iota \# \texttt{op } e\ (y.c)) \rightsquigarrow \\
  & ocs_{ \iota \# \texttt{op }} (e, \lambda y:B^{op} \mapsto \texttt{with } h \texttt{ handle } c)
\end{aligned}
}
\]

% \begin{prooftree}
%     \AxiomC{$$h = \texttt{handler val } x \mapsto c_v \mid ocs$$}
%     \AxiomC{$$(op: A^{op} \rightarrow B^{op}) \in \Sigma_E$$}
%     \BinaryInfC{
%     $$
%     \begin{aligned}
%   & \texttt{with } h \texttt{ handle } (\iota \# \texttt{op } e\ (y.c)) \rightsquigarrow \\
%   & ocs_{ \iota \# \texttt{op }} (e, \lambda y:B^{op} \mapsto \texttt{with } h \texttt{ handle } c)

% \end{aligned}
% $$
% }
% \end{prooftree}

where

\begin{aligned}
& (nil)_{\iota \# \texttt{op }}(e,\kappa) = \iota \# \texttt{op } e (y. \kappa y) \hfill \\
& (\iota' \# \texttt{op' } x k \mapsto c \mid ocs)_{\iota \# \texttt{op }} (e, \kappa) =
\begin{cases}
      c [ e/x, \kappa / k ] & \text{if }\ \iota \# \texttt{op} = \iota' \# \texttt{op' } \\
      ocs_{\iota \# \texttt{op}}(e, \kappa) & \text{otherwise}
    \end{cases}
\end{aligned}

\end{frame}

\begin{frame}{Example: Handling an Operation — \texttt{Choose}}

\textbf{Effect:}
\[
\texttt{effect Choose : unit} \rightarrow \texttt{bool}
\]

\textbf{Computation:}
\[
\texttt{let x = perform (Choose ()) in if x then 1 else 2}
\]

\textbf{Handler:}
\[
\texttt{handler val x} \mapsto [x] \mid
\texttt{Choose } u\ k \mapsto k(\texttt{true}) @ k(\texttt{false})
\]

% \vspace{1em}
\textbf{Operational Step:}
\[
\texttt{with h handle let x = } \iota \# \texttt{Choose () in if x then 1 else 2} 
\rightsquigarrow \\
\texttt{let x = true in if x then 1 else 2} @
\texttt{let x = false in if x then 1 else 2}
\]

% \vspace{1em}
% \textbf{Handler applies the rule:}
% \begin{itemize}
%   \item Matches operation: \( \iota \# \texttt{Choose} \)
%   \item Binds \( e = () \)
%   \item Continuation \( \kappa = \lambda y.\ \texttt{with h handle let x = y in if x then 1 else 2} \)
%   \item Resumes with both \( \texttt{true} \) and \( \texttt{false} \)
% \end{itemize}
\end{frame}

% \begin{frame}{Example 3.1 — Initial Term Under Handler}

% \[
% \\
% \texttt{with } h \texttt{ handle }\\
% \texttt{let } x_1 = \iota \# \texttt{lookup }() \texttt{ in } \\
% \texttt{let } x_2 = \iota \# \texttt{update } x_1 \texttt{ in }
% \texttt{val (succ } x_1)
% \]

% \vspace{1em}
% \textbf{Handler:}
% \[
% \texttt{handler}
% \begin{cases}
% \texttt{val } x \mapsto \iota \# \texttt{update } x \\
% \iota \# \texttt{lookup } x\ k \mapsto k\ 1 \\
% \iota \# \texttt{update } x\ k \mapsto k\ ()
% \end{cases}
% \]

% \vspace{1em}
% \textbf{Goal:} Reduce this computation step-by-step according to the Core Eff small-step semantics.
% \end{frame}

\begin{frame}{Example 3.1 — Initial Configuration}

\textbf{Expression:}
\begin{align*}
&\texttt{with } h\ \texttt{handle let } x_1 = \iota \# \texttt{lookup }() \\
&\quad \texttt{in let } x_2 = \iota \# \texttt{update } x_1 \\
&\quad \texttt{in val (succ } x_1)
\end{align*}

\textbf{Handler:}
\[
\texttt{handler}
\begin{cases}
  \texttt{val } x \mapsto \iota \# \texttt{update } x \\
  \iota \# \texttt{lookup } x\ k \mapsto k\ 1 \\
  \iota \# \texttt{update } x\ k \mapsto k\ ()
\end{cases}
\]
\end{frame}

\begin{frame}{Step 1 — Suspend \texttt{lookup}}

\begin{align*}
&\texttt{let } x_1 = \iota \# \texttt{lookup }() \texttt{ in } c \\
&\rightsquigarrow \iota \# \texttt{lookup }() \left(y_1.\ \texttt{let } x_1 = \texttt{val } y_1 \texttt{ in } c\right)
\end{align*}

Apply to our term:
\begin{align*}
&\texttt{with } h\ \texttt{handle } \iota \# \texttt{lookup }() \Big(y_1. \texttt{let } x_1 = \texttt{val } y_1 \\
&\quad \texttt{in let } x_2 = \iota \# \texttt{update } x_1 \\
&\quad \texttt{in val (succ } x_1)\Big)
\end{align*}
\end{frame}

\begin{frame}{Step 2 — Apply \texttt{lookup} Handler}

Handler clause:
\[
\iota \# \texttt{lookup } x\ k \mapsto k\ 1
\]

Apply:
\begin{align*}
k &= \lambda y_1.\ \texttt{with } h\ \texttt{handle let } x_1 = \texttt{val } y_1 \\
  &\quad \texttt{in let } x_2 = \iota \# \texttt{update } x_1 \\
  &\quad \texttt{in val (succ } x_1)
\end{align*}

\textbf{Substitute 1 for \( y_1 \):}
\begin{align*}
&\texttt{with } h\ \texttt{handle let } x_1 = \texttt{val } 1 \\
&\quad \texttt{in let } x_2 = \iota \# \texttt{update } x_1 \\
&\quad \texttt{in val (succ } x_1)
\end{align*}
\end{frame}


\begin{frame}{Step 3 — Reduce \texttt{update} and \texttt{succ}}

\textbf{Substitute \( x_1 = 1 \):}
\begin{align*}
&\texttt{with } h\ \texttt{handle let } x_2 = \iota \# \texttt{update } 1 \texttt{ in val } 2
\end{align*}

\textbf{Desugar generic effect:}
\begin{align*}
&\equiv \texttt{with } h\ \texttt{handle let } x_2 = \iota \# \texttt{update } 1\ (y_2.\ \texttt{val } y_2) \\
&\quad \texttt{in val } 2
\end{align*}
\end{frame}

 \begin{frame}{Step 4 — Handle Second Operation}

Suspend:
\begin{align*}
&\rightsquigarrow \texttt{with } h\ \texttt{handle } \iota \# \texttt{update } 1 \\
&\quad \left(y_2.\ \texttt{let } x_2 = \texttt{val } y_2 \texttt{ in val } 2\right)
\end{align*}

Handler clause:
\[
\iota \# \texttt{update } x\ k \mapsto k\ ()
\]

Apply:
\begin{align*}
&\rightsquigarrow \texttt{with } h\ \texttt{handle let } x_2 = \texttt{val } () \texttt{ in val } 2 \\
&\rightsquigarrow \texttt{with } h\ \texttt{handle val } 2
\end{align*}
\end{frame}

\begin{frame}{Step 5 — Final Handler Clause Applies}

Handler clause:
\[
\texttt{val } x \mapsto \iota \# \texttt{update } x
\]

Apply:
\[
\texttt{with } h\ \texttt{handle val } 2
\rightsquigarrow
\iota \# \texttt{update } 2
\]

\vspace{1em}
\textbf{Final result:}
\[
\iota \# \texttt{update } 2
\]

This update is not handled — it escapes the scope of the handler.
\end{frame}

\begin{frame}{Big-Step Semantics: Core Judgment}

\textbf{Form of the judgment:}
\[
\langle c, h \rangle \Downarrow r
\]

\textbf{Meaning:}
\begin{itemize}
  \item Under handler \( h \), computation \( c \) evaluates to result \( r \).
  \item This result is either:
    \begin{itemize}
      \item A value: \( \texttt{val } v \)
      \item A suspended operation: \( \iota \# \texttt{op } e (y.c) \)
    \end{itemize}
  \item The handler may handle effects as evaluation proceeds.
\end{itemize}
\end{frame}

\begin{frame}{Big-Step Semantics: Possible Results}

\textbf{Results \( r \) have two forms:}
\[
r ::= \texttt{val } v \quad \mid \quad \iota \# \texttt{op } e (y.c)
\]

\textbf{Interpretation:}
\begin{itemize}
  \item \( \texttt{val } v \): the computation completed with value \( v \)
  \item \( \iota \# \texttt{op } e (y.c) \):
    \begin{itemize}
      \item An operation \( \texttt{op} \) was performed on instance \( \iota \)
      \item \( e \) is the argument
      \item \( y.c \) is the continuation (what happens next)
    \end{itemize}
\end{itemize}
\end{frame}

\begin{frame}{Big-Step Semantics: Rule Format}

Each rule is of the form:
\[
\frac{\text{premises}}{\langle c, h \rangle \Downarrow r}
\]

\textbf{Example:}
\[
\frac{}{\langle \texttt{val } v, h \rangle \Downarrow \texttt{val } v}
\]

This is the base case — returning a value directly.
\end{frame}

% \begin{frame}{Big-Step Semantics: Evaluation Judgment}

% \textbf{Judgment:}
% \[
% c \Downarrow r
% \]

% \textbf{Result \(r\):}
% \[
% r ::= \texttt{val } e \quad \mid \quad \iota \# \texttt{op } e (x. c)
% \]

% \vspace{1em}
% \textbf{Interpretation:}
% \begin{itemize}
%   \item \(c\) evaluates to a value \(e\), or
%   \item suspends on an operation \(\iota \# \texttt{op}\), with continuation \(x.c\)
% \end{itemize}
% \end{frame}

% \begin{frame}{Big-Step Semantics: Base Cases}

% \begin{align*}
% &\frac{}{\texttt{val } e \Downarrow \texttt{val } e} \\
% &\frac{}{\iota \# \texttt{op } e (x. c) \Downarrow \iota \# \texttt{op } e (x. c)}
% \end{align*}

% \vspace{1em}
% \textbf{A value returns itself, and an operation suspends immediately.}
% \end{frame}

\begin{frame}{Big-Step Semantics: Evaluation Judgment}

\textbf{Judgment:}
\[
c \Downarrow r
\]

\textbf{Result \(r\):}
\[
r ::= \texttt{val } e \quad \mid \quad \iota \# \texttt{op } e (x. c)
\]

\vspace{1em}
\textbf{Interpretation:}
\begin{itemize}
  \item \(c\) evaluates to a value \(e\), or
  \item suspends on an operation \(\iota \# \texttt{op}\), with continuation \(x.c\)
\end{itemize}
\end{frame}

\begin{frame}{Big-Step Semantics: Base Cases}

\begin{align*}
&\frac{}{\texttt{val } e \Downarrow \texttt{val } e} \\
&\frac{}{\iota \# \texttt{op } e (x. c) \Downarrow \iota \# \texttt{op } e (x. c)}
\end{align*}

\vspace{1em}
\textbf{A value returns itself, and an operation suspends immediately.}
\end{frame}

\begin{frame}{Big-Step Semantics: Let Binding}

\begin{align*}
&\frac{c_1 \Downarrow \texttt{val } e \quad c_2[e/x] \Downarrow r}
{\texttt{let } x = c_1 \texttt{ in } c_2 \Downarrow r}
\\[1em]
&\frac{c_1 \Downarrow \iota \# \texttt{op } e (y. c)}
{\texttt{let } x = c_1 \texttt{ in } c_2 \Downarrow \iota \# \texttt{op } e (y. \texttt{let } x = c \texttt{ in } c_2)}
\end{align*}
\end{frame}

\begin{frame}{Big-Step Semantics: Conditionals and Matches}

\begin{align*}
&\frac{c_1 \Downarrow r}
{\texttt{if true then } c_1 \texttt{ else } c_2 \Downarrow r}
\\
&\frac{c_2 \Downarrow r}
{\texttt{if false then } c_1 \texttt{ else } c_2 \Downarrow r}
\\[1em]
&\frac{c_1 \Downarrow r}
{\texttt{match } 0 \texttt{ with } 0 \mapsto c_1 \mid \texttt{succ } x \mapsto c_2 \Downarrow r}
\\
&\frac{c_2[e/x] \Downarrow r}
{\texttt{match succ } e \texttt{ with } 0 \mapsto c_1 \mid \texttt{succ } x \mapsto c_2 \Downarrow r}
\end{align*}
\end{frame}

\begin{frame}{Big-Step Semantics: Application and Recursion}

\begin{align*}
&\frac{c[e/x] \Downarrow r}
{(\texttt{fun } x \mapsto c)\ e \Downarrow r}
\\[1em]
&\frac{c_2[(\texttt{fun } x \mapsto \texttt{let rec } f x = c_1 \texttt{ in } c_1)/f] \Downarrow r}
{\texttt{let rec } f x = c_1 \texttt{ in } c_2 \Downarrow r}
\end{align*}
\end{frame}
\begin{frame}{Handler Semantics: Handling \texttt{val}}

\textbf{Handler:}
\[
h = \texttt{handler val } x \mapsto c_v \mid \texttt{op } x\ k \mapsto c_{\text{op}}
\]

\textbf{Rule:}
\[
\frac{c \Downarrow \texttt{val } e \quad c_v[e/x] \Downarrow r}
{\texttt{with } h \texttt{ handle } c \Downarrow r}
\]

\vspace{1em}
\textbf{Interpretation:}
\begin{itemize}
  \item If computation \( c \) evaluates to a value,
  \item we substitute into the handler's \texttt{val} clause and evaluate it.
\end{itemize}
\end{frame}
\begin{frame}{Handler Semantics: Handling an Operation}

\textbf{Handler:}
\[
h = \texttt{handler val } x \mapsto c_v \mid \texttt{op } x\ k \mapsto c_{\text{op}}
\]

\textbf{Rule:}
\[
\frac{
  c \Downarrow \iota \# \texttt{op } e (y. c') \quad
  c_{\text{op}}[e/x,\ \lambda y.\ \texttt{with } h \texttt{ handle } c'/k] \Downarrow r
}{
  \texttt{with } h \texttt{ handle } c \Downarrow r
}
\]

\vspace{1em}
\textbf{Interpretation:}
\begin{itemize}
  \item If \( c \) suspends on an operation matched by \( h \),
  \item we apply the handler’s \texttt{op} clause,
  \item passing the argument \( e \) and a resumable continuation.
\end{itemize}
\end{frame}

\begin{frame}{Handler Semantics: Forwarding Unhandled Operations}

\textbf{Rule (when operation is not handled):}
\[
\frac{
  c \Downarrow \iota \# \texttt{op } e (y. c') \quad
  r = \iota \# \texttt{op } e (y.\ \texttt{with } h \texttt{ handle } c')
}{
  \texttt{with } h \texttt{ handle } c \Downarrow r
}
\]

\vspace{1em}
\textbf{Interpretation:}
\begin{itemize}
  \item If the handler does not match the operation,
  \item we re-emit it, wrapping the continuation with the same handler.
\end{itemize}
\end{frame}

\begin{frame}{Proposition 3.2 — Semantic Equivalence}

\textbf{Statement:}
\[
c \rightsquigarrow^{*} r \quad \Longleftrightarrow \quad c \Downarrow r
\]

\vspace{1em}
\textbf{Explanation:}
\begin{itemize}
  \item \( \rightsquigarrow^{*} \): zero or more steps of the small-step reduction.
  \item \( \Downarrow \): big-step evaluation to result \( r \).
  \item \( r \): either a value \( \texttt{val } v \) or a suspended operation \( \iota \# \texttt{op } e (x. c) \).
\end{itemize}

\vspace{1em}
\textbf{Interpretation:}
\begin{itemize}
  \item Big-step and small-step semantics produce the same results.
  \item This ensures both operational models are equivalent and interchangeable.
\end{itemize}

\end{frame}